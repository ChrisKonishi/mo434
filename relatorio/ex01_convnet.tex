\section{Exercício 1 -- Introdução ao Pytorch Convnet}

\begin{itemize}
	\item Notebook: \textit{introducing-pytorch-convnet.ipynb}
\end{itemize}

\subsection{Descrição}

O notebook em questão contém uma série de demonstrações de conceitos básicos do Pytorch, como as operações de convolução, a função de ativação, \textit{pooling}. Além de \textit{skip connections}, camadas densamente conectadas, \textit{dropout} etc. Por fim, há um processo de construção e treinamento de uma rede neural, desde sua construção, definição de função de perda e otimizador, processo de treinamento e validação.

O problema do processo que foi realizado é que houve \textit{overfitting}. O exercício em questão consiste numa exploração dos elementos e hiperparâmetros utilizados para melhorar o resultado obtido

\subsection{Conceitos}